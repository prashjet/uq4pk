\documentclass{article}


\usepackage[utf8]{inputenc}
\usepackage[T1]{fontenc}
\usepackage{amsmath,amssymb,dsfont}
\numberwithin{equation}{section}
\usepackage{microtype}
\usepackage{graphicx,tikz,pgfplots}
\graphicspath{{images/}}
\pgfplotsset{compat=newest}
\usepackage[hyperref,amsmath,thmmarks]{ntheorem}
\usepackage{aliascnt}
\usepackage[a4paper,centering,bindingoffset=0cm,marginpar=2cm,margin=2.5cm]{geometry}
\usepackage[pagestyles]{titlesec}
\usepackage[font=footnotesize,format=plain,labelfont=sc,textfont=sl,width=0.75\textwidth,labelsep=period]{caption}

\usepackage{bm}
\usepackage{algorithm, algpseudocode}

%'
%' biblatex
%'
\usepackage[backend=biber,maxnames=10,backref=true,hyperref=true,giveninits=true,safeinputenc]{biblatex}
%FINALADDBIB
\addbibresource{journals.bib}%FINALREMOVE
\addbibresource{articles.bib}%FINALREMOVE
\addbibresource{books.bib}%FINALREMOVE
\addbibresource{inproceedings.bib}%FINALREMOVE
\addbibresource{incollection.bib}%FINALREMOVE
\addbibresource{proceedings.bib}%FINALREMOVE
\addbibresource{preprints.bib}%FINALREMOVE
\addbibresource{infmath.bib}%FINALREMOVE
\addbibresource{infmath_books.bib}%FINALREMOVE
\addbibresource{infmath_report.bib}%FINALREMOVE

\DefineBibliographyStrings{english}{%
	backrefpage = {cited on page},
	backrefpages = {cited on pages},
}

\usepackage[pdftex,colorlinks=true,linkcolor=blue,citecolor=green,urlcolor=blue,bookmarks=true,bookmarksnumbered=true]{hyperref}
\hypersetup
{
    pdfauthor={F. Parzer},
    pdfsubject={Subject},
    pdftitle={derivatives},
    pdfkeywords={Keywords}
}
\def\sectionautorefname{Section}
\def\subsectionautorefname{Section}

%'
%' defines the pagestyle (ADAPT)
%'
\newpagestyle{headers}
{
	\headrule
	\sethead[\footnotesize\thepage][\footnotesize\sc Authors][]{}{\footnotesize\sc Title}{\footnotesize\thepage}
	\setfoot{}{}{}
}
\pagestyle{headers}

%'
%' default paragraph layout.
%'
\postdisplaypenalty= 1000
\widowpenalty = 1000
\clubpenalty = 1000
\displaywidowpenalty = 1000
\setlength{\parindent}{0pt}
\setlength{\parskip}{1ex}
\renewcommand{\labelenumi}{\textit{(\roman{enumi})}}
\renewcommand{\theenumi}{\textit{(\roman{enumi})}}

\newtheorem{lemma}{Lemma}[section]
\def\lemmaautorefname{Lemma}

\newaliascnt{proposition}{lemma}
\newtheorem{proposition}[proposition]{Proposition}
\aliascntresetthe{proposition}
\def\propositionautorefname{Proposition}

\newaliascnt{corollary}{lemma}
\newtheorem{corollary}[corollary]{Corollary}
\aliascntresetthe{corollary}
\def\corollaryautorefname{Corollary}

\newaliascnt{theorem}{lemma}
\newtheorem{theorem}[theorem]{Theorem}
\aliascntresetthe{theorem}
\def\theoremautorefname{Theorem}

\newaliascnt{remark}{lemma}
\newtheorem{remark}[remark]{Remark}
\aliascntresetthe{remark}
\def\theoremautorefname{Remark}

\theorembodyfont{\normalfont}
\newaliascnt{definition}{lemma}
\newtheorem{definition}[definition]{Definition}
\aliascntresetthe{definition}
\def\definitionautorefname{Definition}

\newaliascnt{assumption}{lemma}
\newtheorem{assumption}[assumption]{Assumption}
\aliascntresetthe{assumption}
\def\assumptionautorefname{Assumption}

\theoremstyle{nonumberplain}
\theoremseparator{:}
\theoremheaderfont{\normalfont\itshape}

\theoremsymbol{\ensuremath{\square}}
\newtheorem{proof}{Proof}

\usepackage{bbm}
\usepackage{verbatim}
\usepackage{hyperref}


\newcommand{\forward}{\bm {\mathcal G}}
\newcommand{\tv}{\bm \theta_v}
\newcommand{\fcoeff}{\tilde{\bm f}}
\newcommand{\dct}{\bm \Phi}
\newcommand{\ft}[2]{\mathcal F_{#1\to#2}}
\newcommand{\ftvw}{\ft{v}{\omega}}



% Imports my custom definitions
\include{abbreviations}

\graphicspath{{src/}}

\title{Derivatives of the forward operator}
\author{F. Parzer}

\begin{document}

\maketitle

\section{The derivative of the forward operator}

Recall that the forward operator is given by
\begin{align}
\forward(f, \bm \theta_v) = \int_{-\infty}^\infty \frac{1}{1+v/c} S \left( \frac{\lambda}{1 + v/c} ; f \right) L(v;  \bm \theta_v) \d v, \label{eq:fullForward}
\end{align}
where
\begin{align}
S(\lambda; f) = \int_\Theta f(\theta) s(\lambda; \bm\theta) \d \bm\theta.\label{eq:compositeSpectrum}
\end{align}
and
\begin{align*}
L(v, \bm \theta_v) := \normal(v; V,\sigma)\left[\sum_{m=0}^M h_m H_m(\hat v)\right],
\end{align*}
where $\normal(\cdot;0,1)$ denotes the density function of the standard normal distribution, and where furthermore 
\begin{align*}
& \hat v := \frac{v - V}{\sigma},\\
& H_m(x) := \frac{H_m^\mathrm{phys}(x)}{\sqrt{m! 2^m }},\\
& H_m^\mathrm{phys}(x) := (-1)^n e^{x^2} \frac{\d^n}{\d x^n}[\e^{-x^2}],
\end{align*}
and $\bm \theta_v$ is the collection of parameters given by
\begin{align*}
 \bm \theta_v = (V, \sigma, h_0, \ldots, h_M).
\end{align*}

Since we actually use the curvelet coefficients $\fcoeff = \dct \bm f \in \R^N$, our observation operator is
\begin{align*}
\tilde \forward(\fcoeff, \tv) = \forward(\dct^{-1}\fcoeff, \tv),
\end{align*}
where $\dct$ is the corresponding discrete curvelet transform. 

\subsection{Derivative with respect to $\theta_v$}

First, let us compute the derivative with respect to $\bm \theta_v = [V, \sigma, h_0, \ldots, h_M]$. Since $\forward(f, \tv)$ depends on $\tv$ only through $L(v,\tv)$, it suffices to compute $\partial_{\tv} L(v,\tv)$, i.e.
\begin{align*}
\partial_{\tv} L(v, \tv) = \begin{bmatrix}
\partial_V L(v,\tv) & \partial_\sigma L(v,\tv) & \partial_{h_0} L(v,\tv) & \hdots & \partial_{h_M} L(v,\tv)
\end{bmatrix}.
\end{align*}
The derivative with respect to $h_0,\ldots,h_M$ is trivial, given by
\begin{align*}
\partial_{h_m} L(v,\tv) = \normal(v;V,\sigma^2) H_m(\frac{v-V}{\sigma}).
\end{align*}
The derivative with respect to $V$ and $\sigma$ are a little bit harder. The derivatives of the normal probability density function with respect to mean and standard deviation are, respectively,
\begin{align*}
& \partial_V \normal(v; V, \sigma^2) = \frac{v-V}{\sigma^2} \normal(v;V,\sigma^2)\\
\text{and} \quad & \partial_\sigma \normal(v; V, \sigma^2) = \frac{(v-V)^2-\sigma^2}{\sigma^3} \normal(v;V,\sigma^2)
\end{align*}
(see \href{https://www.wolframalpha.com/input/?i=derivative&assumption=%7B%22C%22%2C+%22derivative%22%7D+-%3E+%7B%22Calculator%22%7D&assumption=%7B%22F%22%2C+%22Derivative%22%2C+%22derivativefunction%22%7D+-%3E%22exp%28+-%28v-V%29%5E2%2F%282*s%5E2%29%29+%2F+sqrt%282*pi*s%5E2%29%22&assumption=%7B%22F%22%2C+%22Derivative%22%2C+%22derivativevariable%22%7D+-%3E%22V%22}{here} and \href{https://www.wolframalpha.com/input/?i=derivative&assumption=%7B%22C%22%2C+%22derivative%22%7D+-%3E+%7B%22Calculator%22%7D&assumption=%7B%22F%22%2C+%22Derivative%22%2C+%22derivativefunction%22%7D+-%3E%22exp%28+-%28v-V%29%5E2%2F%282*s%5E2%29%29+%2F+sqrt%282*pi*s%5E2%29%22&assumption=%7B%22F%22%2C+%22Derivative%22%2C+%22derivativevariable%22%7D+-%3E%22s%22}{here}). Finally, the Hermite polynomials satisfy the useful \href{https://en.wikipedia.org/wiki/Hermite_polynomials#Recurrence_relation}{relation}
\begin{align}
H_m'(x) = \sqrt{2m} H_{m-1}(x). \label{eq:hermiteReccurence}
\end{align}
Thus, we obtain the derivatives of $L(v, \tv)$ with respect to $V$ and $\sigma$:
\begin{align*}
& \partial_V L(v; V,\sigma^2) = \frac{v-V}{\sigma^2} - \frac{\sqrt 2}{\sigma} \normal(v; V, \sigma^2) \sum_{m=0}^M h_m \sqrt m H_{m-1}(\frac{v-V}{\sigma}), \\
& \partial_\sigma L(v; V, \sigma^2) = \frac{(v-V)^2-\sigma^2}{\sigma^3} L(v, \tv) - \frac{\sqrt 2 (v-V)}{\sigma^2} \normal(v; V,\sigma^2) \sum_{m=1}^M h_m \sqrt m H_{m-1}(\frac{v-V}{\sigma}).
\end{align*}

\subsection{Derivative of the Fourier transform}

For the actual implementation, we need the derivative of the Fourier transform of $L$. Let $\ftvw L(\omega,\tv)$ denote the Fourier transform of $v \mapsto L(v,\tv)$. Then, it was shown by Cappellari (2016) that
\begin{align*}
\ftvw L(\omega, \tv) = \frac{e^{i\omega V -\sigma^2 \omega^2/2}}{\sqrt{2\pi}}\sum_{m=0}^M i^m h_m H_m(\sigma \omega).
\end{align*}
We can obtain the derivative with respect to $\tv$ similarly as above. First of all, we clearly have
\begin{align*}
\partial_{h_m} \ftvw L(\omega,\tv) = \frac{e^{i\omega V -\sigma^2 \omega^2/2}}{\sqrt{2\pi}} i^m H_m(\sigma \omega), \qquad m = 0,\ldots,M.
\end{align*}
The derivative with respect to $V$ is even easier than in time domain:
\begin{align*}
\partial_V \ftvw L(\omega, \tv) = i \omega \ftvw L(\omega, \tv).
\end{align*}
For the derivative with respect to $\sigma$, we have to use the product rule and \eqref{eq:hermiteReccurence}:
\begin{align*}
\partial_\sigma \ftvw L(\omega, \tv) = - \sigma \omega^2 \ftvw L(\omega, \tv) + \frac{\omega e^{i\omega V -\sigma^2 \omega^2/2}}{\sqrt{\pi}} \sum_{m=1}^M i^m \sqrt m h_m H_{m-1}(\sigma \omega).
\end{align*}

\addcontentsline{toc}{section}{Bibliography}

\printbibliography

\end{document}
